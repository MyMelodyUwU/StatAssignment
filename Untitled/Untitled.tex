\PassOptionsToPackage{unicode=true}{hyperref} % options for packages loaded elsewhere
\PassOptionsToPackage{hyphens}{url}
\documentclass[10pt,ignorenonframetext,aspectratio=169]{beamer}
\setbeamertemplate{caption}[numbered]
\setbeamertemplate{caption label separator}{: }
\setbeamercolor{caption name}{fg=normal text.fg}
\beamertemplatenavigationsymbolsempty
\usepackage{lmodern}
\usepackage{amssymb,amsmath}
% Thanks, @Xyv
\usepackage{calc}
\usepackage{ifxetex,ifluatex}
\usepackage{fixltx2e} % provides \textsubscript
\ifnum 0\ifxetex 1\fi\ifluatex 1\fi=0 % if pdftex
  \usepackage[T1]{fontenc}
  \usepackage[utf8]{inputenc}
\else % if luatex or xelatex
  \ifxetex
    \usepackage{mathspec}
  \else
    \usepackage{fontspec}
\fi
\defaultfontfeatures{Ligatures=TeX,Scale=MatchLowercase}







\fi

  \usetheme[]{macquarie}



% A default size of 24 is set in beamerthememonash.sty




% use upquote if available, for straight quotes in verbatim environments
\IfFileExists{upquote.sty}{\usepackage{upquote}}{}
% use microtype if available
\IfFileExists{microtype.sty}{%
  \usepackage{microtype}
  \UseMicrotypeSet[protrusion]{basicmath} % disable protrusion for tt fonts
}{}


\newif\ifbibliography


\hypersetup{
      pdftitle={STAT 1378: A Thomas Fung Appreciation Society},
            pdfborder={0 0 0},
    breaklinks=true}
%\urlstyle{same}  % Use monospace font for urls




  \usepackage{color}
  \usepackage{fancyvrb}
  \newcommand{\VerbBar}{|}
  \newcommand{\VERB}{\Verb[commandchars=\\\{\}]}
  \DefineVerbatimEnvironment{Highlighting}{Verbatim}{commandchars=\\\{\}}
  % Add ',fontsize=\small' for more characters per line
  \usepackage{framed}
  \definecolor{shadecolor}{RGB}{248,248,248}
  \newenvironment{Shaded}{\begin{snugshade}}{\end{snugshade}}
  \newcommand{\AlertTok}[1]{\textcolor[rgb]{0.94,0.16,0.16}{#1}}
  \newcommand{\AnnotationTok}[1]{\textcolor[rgb]{0.56,0.35,0.01}{\textbf{\textit{#1}}}}
  \newcommand{\AttributeTok}[1]{\textcolor[rgb]{0.77,0.63,0.00}{#1}}
  \newcommand{\BaseNTok}[1]{\textcolor[rgb]{0.00,0.00,0.81}{#1}}
  \newcommand{\BuiltInTok}[1]{#1}
  \newcommand{\CharTok}[1]{\textcolor[rgb]{0.31,0.60,0.02}{#1}}
  \newcommand{\CommentTok}[1]{\textcolor[rgb]{0.56,0.35,0.01}{\textit{#1}}}
  \newcommand{\CommentVarTok}[1]{\textcolor[rgb]{0.56,0.35,0.01}{\textbf{\textit{#1}}}}
  \newcommand{\ConstantTok}[1]{\textcolor[rgb]{0.00,0.00,0.00}{#1}}
  \newcommand{\ControlFlowTok}[1]{\textcolor[rgb]{0.13,0.29,0.53}{\textbf{#1}}}
  \newcommand{\DataTypeTok}[1]{\textcolor[rgb]{0.13,0.29,0.53}{#1}}
  \newcommand{\DecValTok}[1]{\textcolor[rgb]{0.00,0.00,0.81}{#1}}
  \newcommand{\DocumentationTok}[1]{\textcolor[rgb]{0.56,0.35,0.01}{\textbf{\textit{#1}}}}
  \newcommand{\ErrorTok}[1]{\textcolor[rgb]{0.64,0.00,0.00}{\textbf{#1}}}
  \newcommand{\ExtensionTok}[1]{#1}
  \newcommand{\FloatTok}[1]{\textcolor[rgb]{0.00,0.00,0.81}{#1}}
  \newcommand{\FunctionTok}[1]{\textcolor[rgb]{0.00,0.00,0.00}{#1}}
  \newcommand{\ImportTok}[1]{#1}
  \newcommand{\InformationTok}[1]{\textcolor[rgb]{0.56,0.35,0.01}{\textbf{\textit{#1}}}}
  \newcommand{\KeywordTok}[1]{\textcolor[rgb]{0.13,0.29,0.53}{\textbf{#1}}}
  \newcommand{\NormalTok}[1]{#1}
  \newcommand{\OperatorTok}[1]{\textcolor[rgb]{0.81,0.36,0.00}{\textbf{#1}}}
  \newcommand{\OtherTok}[1]{\textcolor[rgb]{0.56,0.35,0.01}{#1}}
  \newcommand{\PreprocessorTok}[1]{\textcolor[rgb]{0.56,0.35,0.01}{\textit{#1}}}
  \newcommand{\RegionMarkerTok}[1]{#1}
  \newcommand{\SpecialCharTok}[1]{\textcolor[rgb]{0.00,0.00,0.00}{#1}}
  \newcommand{\SpecialStringTok}[1]{\textcolor[rgb]{0.31,0.60,0.02}{#1}}
  \newcommand{\StringTok}[1]{\textcolor[rgb]{0.31,0.60,0.02}{#1}}
  \newcommand{\VariableTok}[1]{\textcolor[rgb]{0.00,0.00,0.00}{#1}}
  \newcommand{\VerbatimStringTok}[1]{\textcolor[rgb]{0.31,0.60,0.02}{#1}}
  \newcommand{\WarningTok}[1]{\textcolor[rgb]{0.56,0.35,0.01}{\textbf{\textit{#1}}}}

  \usepackage{longtable,booktabs}
  \usepackage{caption}
  % These lines are needed to make table captions work with longtable:
  \makeatletter
  \def\fnum@table{\tablename~\thetable}
  \makeatother


% From {rticles}
\newlength{\csllabelwidth}
\setlength{\csllabelwidth}{3em}
\newlength{\cslhangindent}
\setlength{\cslhangindent}{1.5em}
% for Pandoc 2.8 to 2.10.1
\newenvironment{cslreferences}%
  {}%
  {\par}
% For Pandoc 2.11+
\newenvironment{CSLReferences}[3] % #1 hanging-ident, #2 entry spacing
 {% don't indent paragraphs
  \setlength{\parindent}{0pt}
  % turn on hanging indent if param 1 is 1
  \ifodd #1 \everypar{\setlength{\hangindent}{\cslhangindent}}\ignorespaces\fi
  % set entry spacing
  \ifnum #2 > 0
  \setlength{\parskip}{#2\baselineskip}
  \fi
 }%
 {}
\usepackage{calc} % for calculating minipage widths
\newcommand{\CSLBlock}[1]{#1\hfill\break}
\newcommand{\CSLLeftMargin}[1]{\parbox[t]{\csllabelwidth}{#1}}
\newcommand{\CSLRightInline}[1]{\parbox[t]{\linewidth - \csllabelwidth}{#1}}
\newcommand{\CSLIndent}[1]{\hspace{\cslhangindent}#1}


  \usepackage{graphicx,grffile}
  \makeatletter
  \def\maxwidth{\ifdim\Gin@nat@width>\linewidth\linewidth\else\Gin@nat@width\fi}
  \def\maxheight{\ifdim\Gin@nat@height>\textheight0.8\textheight\else\Gin@nat@height\fi}
  \makeatother
  % Scale images if necessary, so that they will not overflow the page
  % margins by default, and it is still possible to overwrite the defaults
  % using explicit options in \includegraphics[width, height, ...]{}
  \setkeys{Gin}{width=\maxwidth,height=\maxheight,keepaspectratio}

% Prevent slide breaks in the middle of a paragraph:
\widowpenalties 1 10000
\raggedbottom

  \AtBeginPart{
    \let\insertpartnumber\relax
    \let\partname\relax
    \frame{\partpage}
  }
  \AtBeginSection{
    \ifbibliography
    \else
      \let\insertsectionnumber\relax
      \let\sectionname\relax
      \frame[plain]{\sectionpage}
    \fi
  }
  \AtBeginSubsection{
    \let\insertsubsectionnumber\relax
      \let\subsectionname\relax
      \frame[plain]{\sectionpage}
  }



\setlength{\parindent}{0pt}
\setlength{\parskip}{6pt plus 2pt minus 1pt}
\setlength{\emergencystretch}{3em}  % prevent overfull lines
\providecommand{\tightlist}{%
  \setlength{\itemsep}{0pt}\setlength{\parskip}{0pt}}

  \setcounter{secnumdepth}{0}




% Redefine shaded environment if it exists (to ensure text is black)
\ifcsname Shaded\endcsname
  \definecolor{shadecolor}{RGB}{225,225,225}
  \renewenvironment{Shaded}{\color{black}\begin{snugshade}\color{black}}{\end{snugshade}}
\fi
%%

  \title[]{STAT 1378: A Thomas Fung Appreciation Society}

  \subtitle{Assignment 3}



\date[
      28 October 2021
  ]{
      28 October 2021
        }

\begin{document}

% Hide progress bar and footline on titlepage
  \begin{frame}[plain]
  \titlepage
  \end{frame}



\hypertarget{intro}{%
\section{Intro}\label{intro}}

\begin{frame}[fragile]{Slide with bullets}
\protect\hypertarget{slide-with-bullets}{}
\begin{itemize}[<+->]
\tightlist
\item
  Bullet 1
\item
  Bullet 2
\item
  Bullet 3
\end{itemize}

Use \texttt{\textbackslash{}alert} to \alert{highlight} some text

\begin{block}{Some enumeration}
\protect\hypertarget{some-enumeration}{}
\begin{enumerate}[<+->]
\tightlist
\item
  The first item
\item
  Stuff
\item
  Nonsense
\end{enumerate}
\end{block}
\end{frame}

\hypertarget{using-r}{%
\section{Using R}\label{using-r}}

\begin{frame}[fragile]{Slide with R output}
\protect\hypertarget{slide-with-r-output}{}
\begin{Shaded}
\begin{Highlighting}[]
\NormalTok{  plot1 }\OtherTok{\textless{}{-}}\NormalTok{ passwords }\SpecialCharTok{\%\textgreater{}\%}
  \FunctionTok{filter}\NormalTok{(category }\SpecialCharTok{==} \StringTok{"animal"}\NormalTok{)}
  \FunctionTok{ggplot}\NormalTok{(}\AttributeTok{data =}\NormalTok{ plot1) }\SpecialCharTok{+}
  \FunctionTok{geom\_line}\NormalTok{(}\AttributeTok{mapping =} \FunctionTok{aes}\NormalTok{(}\AttributeTok{x =}\NormalTok{ strength, }\AttributeTok{y =}\NormalTok{ offline\_crack\_sec)) }\SpecialCharTok{+}
  \FunctionTok{facet\_wrap}\NormalTok{(}\SpecialCharTok{\textasciitilde{}}\NormalTok{plot1}\SpecialCharTok{$}\NormalTok{password, }\AttributeTok{scales =} \StringTok{"free"}\NormalTok{) }\SpecialCharTok{+}
  \FunctionTok{labs}\NormalTok{(}\AttributeTok{title =} \StringTok{"Plot \#1: How the Three Lap (with no shortcut) develops overtime"}\NormalTok{)}
\end{Highlighting}
\end{Shaded}

\includegraphics{Untitled_files/figure-beamer/p1-1.pdf}

\begin{Shaded}
\begin{Highlighting}[]
\NormalTok{  plot1}
\end{Highlighting}
\end{Shaded}

\begin{verbatim}
## # A tibble: 29 x 9
##     rank password category value time_unit offline_crack_sec rank_alt strength
##    <dbl> <chr>    <chr>    <dbl> <chr>                 <dbl>    <dbl>    <dbl>
##  1     7 dragon   animal    3.72 days                0.00321        7        8
##  2    11 monkey   animal    3.72 days                0.00321       11        8
##  3    90 bigdog   animal    3.72 days                0.00321       90        7
##  4    98 falcon   animal    3.72 days                0.00321       98        8
##  5   105 phoenix  animal    3.19 months              0.0835       105        9
##  6   108 tigers   animal    3.72 days                0.00321      108        8
##  7   131 chicken  animal    3.19 months              0.0835       131        8
##  8   144 bulldog  animal    3.19 months              0.0835       144        7
##  9   160 spider   animal    3.72 days                0.00321      160        8
## 10   163 rabbit   animal    3.72 days                0.00321      163        6
## # ... with 19 more rows, and 1 more variable: font_size <dbl>
\end{verbatim}
\end{frame}

\begin{frame}[fragile]{Slide with graphics}
\protect\hypertarget{slide-with-graphics}{}
\begin{Shaded}
\begin{Highlighting}[]
\NormalTok{plot2 }\OtherTok{\textless{}{-}}\NormalTok{ passwords }\SpecialCharTok{\%\textgreater{}\%}
    \FunctionTok{filter}\NormalTok{(time\_unit }\SpecialCharTok{==} \StringTok{"years"}\NormalTok{) }\SpecialCharTok{\%\textgreater{}\%}
    \FunctionTok{group\_by}\NormalTok{(}\FunctionTok{interaction}\NormalTok{(strength,offline\_crack\_sec)) }\SpecialCharTok{\%\textgreater{}\%}
    \CommentTok{\# mutate(Banana = paste(type,ifelse(shortcut == "Yes", "With Shourtcut","Without Shortcut"))) \%\textgreater{}\%}
    \CommentTok{\# ggplot(aes(x = date, y = time, color = Race)) +}
    \FunctionTok{ggplot}\NormalTok{(}\FunctionTok{aes}\NormalTok{(}\AttributeTok{x =}\NormalTok{ strength, }\AttributeTok{y =}\NormalTok{ offline\_crack\_sec)) }\SpecialCharTok{+}
    \FunctionTok{geom\_point}\NormalTok{() }\SpecialCharTok{+}
    \FunctionTok{geom\_line}\NormalTok{() }\SpecialCharTok{+}
    \FunctionTok{labs}\NormalTok{(}\AttributeTok{title =} \StringTok{"A strength vs time crack chart"}\NormalTok{, }\AttributeTok{subtitle =} \StringTok{"?"}\NormalTok{, }\AttributeTok{x =} \StringTok{"Strength"}\NormalTok{, }\AttributeTok{y =} \StringTok{"crack time"}\NormalTok{)}

\NormalTok{plot2}
\end{Highlighting}
\end{Shaded}

\begin{center}\includegraphics{Untitled_files/figure-beamer/plot2-1} \end{center}

\begin{Shaded}
\begin{Highlighting}[]
\CommentTok{\# i want a histogram here}
\CommentTok{\# i want advice on what i should look for in the histogram}
\end{Highlighting}
\end{Shaded}
\end{frame}

\begin{frame}{Slide with mathematics}
\protect\hypertarget{slide-with-mathematics}{}
Suppose \(X_1, X_2, \ldots, X_n\) are independent and identitically
distributed random variables with common cumulative distribution
function \(F_X\) with support on \(\mathbb{R}\). The empirical
cumulative distribution function is defined with, \[
  F_n(x) = \frac{1}{n}\sum_{i = 1}^n I_{\left( -\infty, x\right]}(X_i)
\] where \(I_A(x)\) denotes the indicator function for the set \(A\).
The following theorem provides uniform covergence for \(F_n\)

\begin{block}{Glivenko-Cantelli Theorem}
\protect\hypertarget{glivenko-cantelli-theorem}{}
If \(X_i\) are i.i.d. with common cdf \(F\) then, \[
  \left\| F_n - F \right\| = \sup_{x \in \mathbb{R}} |F_n(x) - F(x)| \stackrel{n \to \infty}{\longrightarrow} 0\quad \text{almost surely.}
\]
\end{block}

\begin{itemize}[<+->]
\tightlist
\item
  See Vaart and Wellner (1996) for more information on Empirical
  processes.
\end{itemize}
\end{frame}

\begin{frame}
A slide with no header if you need more space.
\end{frame}

\hypertarget{rmarkdown-examples}{%
\section{RMarkdown Examples}\label{rmarkdown-examples}}

\begin{frame}[fragile]{R Figure}
\protect\hypertarget{r-figure}{}
\begin{Shaded}
\begin{Highlighting}[]
\FunctionTok{plot}\NormalTok{(mpg }\SpecialCharTok{\textasciitilde{}}\NormalTok{ cyl, }\AttributeTok{data =}\NormalTok{ mtcars)}
\end{Highlighting}
\end{Shaded}

\includegraphics{Untitled_files/figure-beamer/mtcarsboxplot-1.pdf}
\end{frame}

\begin{frame}[fragile]{R Table}
\protect\hypertarget{r-table}{}
A simple \texttt{knitr::kable} example:

\small

\begin{Shaded}
\begin{Highlighting}[]
\NormalTok{knitr}\SpecialCharTok{::}\FunctionTok{kable}\NormalTok{(}\FunctionTok{head}\NormalTok{(mtcars),}
       \AttributeTok{caption=}\StringTok{"First few observations of the mtcars dataset"}\NormalTok{)}
\end{Highlighting}
\end{Shaded}

\begin{longtable}[]{@{}lrrrrrrrrrrr@{}}
\caption{First few observations of the mtcars dataset}\tabularnewline
\toprule
& mpg & cyl & disp & hp & drat & wt & qsec & vs & am & gear & carb \\
\midrule
\endfirsthead
\toprule
& mpg & cyl & disp & hp & drat & wt & qsec & vs & am & gear & carb \\
\midrule
\endhead
Mazda RX4 & 21.0 & 6 & 160 & 110 & 3.90 & 2.620 & 16.46 & 0 & 1 & 4 &
4 \\
Mazda RX4 Wag & 21.0 & 6 & 160 & 110 & 3.90 & 2.875 & 17.02 & 0 & 1 & 4
& 4 \\
Datsun 710 & 22.8 & 4 & 108 & 93 & 3.85 & 2.320 & 18.61 & 1 & 1 & 4 &
1 \\
Hornet 4 Drive & 21.4 & 6 & 258 & 110 & 3.08 & 3.215 & 19.44 & 1 & 0 & 3
& 1 \\
Hornet Sportabout & 18.7 & 8 & 360 & 175 & 3.15 & 3.440 & 17.02 & 0 & 0
& 3 & 2 \\
Valiant & 18.1 & 6 & 225 & 105 & 2.76 & 3.460 & 20.22 & 1 & 0 & 3 & 1 \\
\bottomrule
\end{longtable}
\end{frame}

\begin{frame}{Resources}
\protect\hypertarget{resources}{}
\begin{itemize}[<+->]
\tightlist
\item
  See the \href{https://github.com/rstudio/rmarkdown}{RMarkdown
  repository} for more on RMarkdown
\item
  Also the
\end{itemize}
\end{frame}

\begin{frame}{References}
\protect\hypertarget{references}{}
\hypertarget{refs}{}
\begin{CSLReferences}{1}{0}
\leavevmode\hypertarget{ref-vanderVaart.Wellner:96}{}%
Vaart, Aad W. van der, and Jon A. Wellner. 1996. \emph{Weak Convergence
and Empirical Processes}. Springer Series in Statistics.
Springer-Verlag, New York.
\url{https://doi.org/10.1007/978-1-4757-2545-2}.

\end{CSLReferences}
\end{frame}




\end{document}
